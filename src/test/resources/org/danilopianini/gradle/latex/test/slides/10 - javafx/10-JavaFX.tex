%\documentclass[12pt,handout]{beamer}
\documentclass[presentation]{beamer}
\usepackage{../oop-slides-croatti}
\usepackage{soul}
\setbeamertemplate{bibliography item}[text]

\newcommand{\lab}{Lab10}

\title[{\lab} -- JavaFX]{Introduzione a JavaFX}

\date[\today]{\today}

\begin{document}

\frame[label=coverpage]{\titlepage}

%====================
%Outline
%====================
\begin{frame}<beamer>
 	\frametitle{Outline}
 	\tableofcontents[]
\end{frame}

%\section{Preparazione dell'ambiente di lavoro}

%\begin{frame}{Preparazione dell'ambiente di lavoro}
%	\begin{enumerate}\itemsep20pt
%		\item Clonare il repository degli esercizi
%		\begin{itemize}
%			\item \url{https://bitbucket.org/danysk/courses-2018-oop-lab-10}
%		\begin{itemize}
%		\item Opzionalmente, si fork-i il repository
%		\end{itemize}
%			\item Si copi la URI con cui effettuare il clone dall'interfaccia web di Bitbucket
%		\end{itemize}
%		\item Importare il repository in Eclipse come progetto Java
%		\item Configurare correttamente Eclipse, abilitare e configurare i plugin per il controllo di qualità del codice
%	\end{enumerate}
%\end{frame}
%
%\begin{frame}{Modalità di lavoro}
%	\begin{enumerate}
%		\item Seguire le istruzioni del file README.md nella root del repository
%		\item Tentare di capire l'esercizio in autonomia
%		\begin{itemize}
%			\item Contattare il docente se qualcosa non è chiaro
%		\end{itemize}
%		\item Risolvere l'esercizio in autonomia
%		\begin{itemize}
%			\item Contattare il docente se si rimane bloccati
%		\end{itemize}
%		\item Utilizzare le funzioni di test per verificare la soluzione realizzata
%		\item Cercare di risolvere autonomamente eventuali piccoli problemi che possono verificarsi durante lo svolgimento degli esercizi
%		\begin{itemize}
%			\item Contattare il docente se, anche \textbf{dopo aver usato il debugger}, non si è riusciti a risalire all'origine del problema
%		\end{itemize}
%		\item Scrivere la Javadoc per l'esercizio svolto
%		\item Assicurarsi che non ci siano warning nel proprio codice
%		\item Effettuare \textit{almeno} un commit ad esercizio completato
%		\item \textbf{A esercizio ultimato sottoporre la soluzione al docente}
%		\item Proseguire con l'esercizio seguente
%	\end{enumerate}
%\end{frame}
%
%\section{Swing Toolkit (richiami)}
%
%\begin{frame}{GUI in Java: Swing Toolkit}
%\begin{itemize}\itemsep10pt
%\item Sviluppato per fornire un insieme di componenti per la realizzazione di GUI, più completi e sofisticati rispetto a quelli presenti in AWT (Abstract Window Toolkit)
%\begin{itemize}
%\item Oltre ai componenti già presenti in AWT, ne propone di nuovi: es. Tabbed Panel, Lists, etc.
%\item Tutti i componenti di swing sono sviluppati in Java (in AWT si utilizza codice nativo) e sono quindi platform-independent (\textit{Lightweight UI})
%\end{itemize}
%\item Propone un proprio look-and-feel per le applicazioni desktop
%\begin{itemize}
%\item Supporta anche i pluggable look-and-feel
%\end{itemize}
%\item Ha un'architettura basata sul pattern MVC e segue un modello di programmazione single-thread
%\end{itemize}
%\end{frame}

\section{Java Swing: Richiami}

\begin{frame}{Event Dispatch Thread (EDT) in Swing}
\begin{itemize}\itemsep10pt
\item E' il thread deputato alla gestione degli eventi della GUI di un'applicazione Swing-based
\item Avviato dalla JVM alla creazione del primo JFrame
\begin{itemize}
\item (nota) L'applicazione non termina al completamento del main
\end{itemize}
\item Tutti gli eventi relativi a componenti della GUI sono demandati all'EDT
\begin{itemize}
\item Ciascun evento è gestito a condizione che la gestione di tutti quelli precedenti sia terminata
\item Si presuppone che la gestione di ciascun evento non implichi una situazione di \emph{stallo} dell'applicazione (dell'EDT)
\end{itemize}
\item \textbf{Swing non è thread-safe!}
\begin{itemize}
\item Si presuppone che l'EDT sia l'unico thread ad eccedere alla GUI
\item Si utilizza la libreria \texttt{SwingUtilities} -- i metodi \texttt{invokeLater()} and \texttt{invokeAndWait()} -- per accedere alla GUI da altri thread
\end{itemize}
\end{itemize}
\end{frame}

\begin{frame}{Limiti (e svantaggi) di Swing}
\begin{itemize}\itemsep10pt
\item L'insieme dei componenti offerti è piuttosto limitato
\begin{itemize}
\item \dots se non altro, per le applicazioni moderne
\item \dots il look-and-feel è old-style
\item \dots non supportano (tutte) le nuove interfacce touch screen
\end{itemize}
\item La gestione dello z-level tra i componenti non è banale
\item Non supporta forme di validazione per i dati nei componenti di input
\item Non propone alcun supporto nativo per la gestione del 3D
\item Ha il limite (e la problematica) della non thread-safeness
\item Non è possibile separare il controller dalla relativa view per i componenti Swing preesistenti
\end{itemize}
\end{frame}

\section{Introduzione a JavaFX}

\begin{frame}{JavaFX}
\begin{itemize}\itemsep20pt
\item Libreria Java per la creazione di GUI per Rich Applications multi-piattaforma
\begin{itemize}
\item Disponibile dal 2008 (v. 1.0 -- 2.2) come libreria stand-alone
\item Presente ``\emph{stabilmente}'' nel JDK da Java 8 (v. JavaFX 8)
\item \st{Introdotto ufficialmente in Java con l'idea di sostituire (gradualmente) Swing}
\item Torna ad essere una libreria stand-alone da Java 11 
\end{itemize}
\item Propone un look-and-feel personalizzabile
\begin{itemize}
\item La descrizione dello stile/aspetto dei componenti della GUI è separato dalla relativa implementazione
\item Segue il pattern MVC
\end{itemize}
\item Consente la creazione di GUI moderne, di qualità e ben adattabili a qualunque piattaforma e supporto hardware
\end{itemize}
\end{frame}

\begin{frame}{JavaFX (dal 2018)}

\begin{block}{Java Client Roadmap Update (Oracle White Paper, March 2018)}
\begin{small}
\begin{center}
\emph{Over the last decade, the JavaFX technology has found its niche where it enjoys the support of a developer community. At the same time, the magnitude of opportunities for cross-platform toolkits [...] has been eroded by the rise of ``mobile first'' and ``web first'' apps.}
\begin{tiny}
\url{https://www.oracle.com/technetwork/java/javase/javaclientroadmapupdate2018mar-4414431.pdf}
\end{tiny}
\end{center}
\end{small}
\end{block}
%
\begin{itemize}
\item Pertanto:
\begin{itemize}
\item JavaFX continuerà ad essere supportato fino al 2022 da Oracle (solo nel JDK 8)
\item JavaFX sara disponibile come libreria esterna opensource (affidata ad OpenJDK) -- \url{https://openjfx.io}
\end{itemize}
\end{itemize}
\end{frame}

%\begin{frame}{JavaFX 11}
%\begin{itemize}\itemsep10pt
%\item Al momento, si tratta del porting as-is di JavaFX 8 sul JDK 11
%\begin{itemize}
%\item Unica novità significativa, FX Robot API per la simulazione dell'interazione dell'utente con la GUI
%\end{itemize}
%\item Implica l'utilizzo del JDK 11
%\item Risente ancora di molti bug 
%\begin{itemize}
%\item \url{https://github.com/javafxports/openjdk-jfx/blob/jfx-11/doc-files/release-notes-11.0.1.md\#release-notes-for-javafx-1101}
%\end{itemize}
%\end{itemize}
%\end{frame}

\subsection{Architettura e Key Features}

\begin{frame}{JavaFX: Key Features (1/2)}
\begin{block}{Java APIs}
\begin{itemize}
\item Libreria che include classi e interfacce scritte in Java e compilato con retro compatibilità fino a Java 7
\end{itemize}
\end{block}
%
\begin{block}{FXML}
\begin{itemize}
\item FXML è un linguaggio dichiarativo per definire la GUI di un'applicazione JavaFX-based
\item Il suo impiego non è indispensabile, ma fortemente consigliato per una buona separazione dei contenuti
\end{itemize}
\end{block}
%
\begin{block}{Interoperabilità con Swing}
\begin{itemize}
\item GUI Swing esistenti possono essere aggiornate con le caratteristiche offerte da JavaFX
\item E' possibile inserire componenti Swing in interfacce JavaFX (si veda la classe \texttt{SwingNode})
\end{itemize}
\end{block}
\end{frame}

\begin{frame}{JavaFX: Key Features (2/2)}
\begin{block}{Graphics API}
\begin{itemize}
\item Supporto nativo per la grafica 3D
\item Abilita la possibilità di disegnare direttamente sulla superficie (canvas) dell'applicazione
\end{itemize}
\end{block}
%
\begin{block}{Supporto per schermi Multi-touch e Hi-DPI}
\begin{itemize}
\item Fornisce il supporto per funzionalità  multi-touch, in funzione della piattaforma in cui l'applicazione è in esecuzione
\item Garantisce una buona visualizzazione della GUI anche su schermi ad alta densità
\end{itemize}
\end{block}
\end{frame}

%\begin{frame}{JavaFX: Architettura}
%\begin{figure}
%\includegraphics[width=0.95\textwidth]{img/javafx-architecture.png}
%\end{figure}
%\begin{itemize}
%\item \emph{per approfondimenti} -- \url{https://docs.oracle.com/javase/8/javafx/get-started-tutorial/jfx-architecture.htm}
%\end{itemize}
%\end{frame}

\subsection{Key Concepts}

\begin{frame}{Elementi fondamentali (1/3)}
\begin{block}{Stage}
\begin{itemize}
\item Il contenitore (esterno) dove la GUI sarà visualizzata
\begin{itemize}
\item es. una finestra del S.O.
\item Equivalente al \texttt{JFrame} di Swing
\item Non è compito del programmatore creare una sua istanza.
\end{itemize}
\item \url{https://openjfx.io/javadoc/13/javafx.graphics/javafx/stage/Stage.html}
\end{itemize}
\end{block}

\begin{block}{Scene}
\begin{itemize}
\item Rappresenta il contenuto di uno Stage (una \emph{pagina} della GUI)
\begin{itemize}
\item ogni Stage può avere più istanze diverse di Scene
\end{itemize}
\item Di fatto, è un container di Node(s)
\item \url{https://openjfx.io/javadoc/13/javafx.graphics/javafx/scene/Scene.html}
\end{itemize}
\end{block}
\end{frame}

\begin{frame}[fragile]{Applicazione JavaFX: GUI vuota}
\begin{lstlisting}
public class Main extends Application {

	@Override
	public void start(Stage stage) throws Exception {
		Group root = new Group();
		Scene scene = new Scene(root, 500, 300);
		
		stage.setTitle("JavaFX Demo");
		stage.setScene(scene);
		stage.show();
	}
	
	public static void main(String[] args) {
		launch(args);
	}
}
\end{lstlisting}
\end{frame}

\begin{frame} {Elementi fondamentali (2/3)}
\begin{block}{Node(s)}
\begin{itemize}
\item Un elemento/componente della scena
\begin{itemize}
\item Ciascun nodo ha sia la parte di view (aspetto) sia la parte di controller (comportamento)
\item Hanno proprietà e possono generare eventi
\end{itemize}
\item Possono essere organizzati gerarchicamente
\item \url{https://openjfx.io/javadoc/13/javafx.graphics/javafx/scene/Node.html}
\end{itemize}
\end{block}
\end{frame}

\begin{frame}{Elementi fondamentali (3/3)}
\begin{figure}
\includegraphics[width=\textwidth]{img/javafx-app.png}
\end{figure}
\end{frame}

%\begin{frame}{Nodes Hiearchy}
%
%\end{frame}

\begin{frame}{Struttura di un'applicazione JavaFX-based}
\begin{figure}
\includegraphics[width=0.95\textwidth]{img/javafx-app-arch.png}
\end{figure}
\end{frame}

\begin{frame}{Linee guida}
\begin{enumerate}\itemsep20pt
\item La classe principale di un'applicazione JavaFX deve estendere la classe \texttt{javafx.application.Application}
\item Il metodo \texttt{main()} deve chiamare il metodo \texttt{launch()}
\begin{itemize}
\item Si tratta di un metodo statico della classe \texttt{Application}
\end{itemize}
\item Il metodo \texttt{void start(Stage stage)} è, di fatto, l'entry point dell'applicazione JavaFX
\item La scena definita per lo stage (vedi metodo \texttt{setScene()}) costituisce il container principale per tutti i componenti della GUI
\end{enumerate}
\end{frame}

\begin{frame}{Nodi e Proprietà}
\begin{itemize}\itemsep10pt
\item Ogni scena può essere popolata con una gerarchia di nodi
\item Ciascun nodo (componente) espone diverse proprietà
\begin{itemize}
\item relative all'aspetto (es. \texttt{size}, \texttt{posizion}, \texttt{color}, \dots)
\item relative al contenuto (es. \texttt{text}, \texttt{value}, \dots)
\item relative al comportamento (es. \texttt{event handlers}, \texttt{controller}, \dots)
\end{itemize}
\item Ciascun nodo genera eventi in relazione ad azioni dell'utente
\end{itemize}
\end{frame}

\begin{frame}[fragile]{GUI con bottone e label}
\begin{lstlisting}
public class Example1 extends Application{
	@Override
	public void start(Stage stage) throws Exception {
		Label lbl = new Label();
		lbl.setText("Label text here...");
		
		Button btn = new Button();
		btn.setText("Click me");
		
		HBox root = new HBox();
		root.getChildren().add(btn);
		root.getChildren().add(lbl);
				
		stage.setTitle("JavaFX - Example 1");
		stage.setScene(new Scene(root, 300,250));
		stage.show();
	}
	
	public static void main(String[] args) { launch(args); }
}
\end{lstlisting}
\end{frame}

\begin{frame}{Layouts (1/2)}
\begin{block}{Group}
\begin{itemize}
\item Non impone nessun posizionamento per i componenti figli
\item Da utilizzare per posizionare i componenti figli in posizioni fisse
\end{itemize}
\end{block}

\begin{block}{Region}
\begin{itemize}
\item Tutte le sue specializzazioni forniscono diversi layout general purpose 
\item Sono simili a quelli offerti da Swing
\end{itemize}
\end{block}

\begin{block}{Control}
\begin{itemize}
\item Costituisce l'insieme dei layout personalizzabili
\item Ciascun layout di questo tipo fornisce specifiche API per l'aggiunta dei componenti figli
\end{itemize}
\end{block}
\url{https://openjfx.io/javadoc/13/javafx.graphics/javafx/scene/layout/package-summary.html}
\end{frame}

\begin{frame}{Layouts (2/2)}
\begin{figure}
\includegraphics[width=0.75\textwidth]{img/layouts.png}
\end{figure}

\begin{block}{Aggiungere componenti ad un layout}
\begin{itemize}
\item Il metodo \texttt{ObservableList<Node> getChildren()} restituisce la lista di nodi figli di un qualunque nodo/layout
\item Alla lista possono essere aggiunti (\texttt{boolean add(Node e)}) e gestiti i componenti figli
\end{itemize}
\end{block}
\end{frame}

\begin{frame}[fragile]{Eventi}
\begin{itemize}\itemsep10pt
\item Possono essere generati in relazione nodi e alle scene
\begin{itemize}
\item Fanno riferimento alla classe \texttt{javafx.event.Event}
\end{itemize}
\item Come in swing, si generano in funzione di azioni dell'utente sulla GUI
\item Possono essere gestiti attraverso \emph{event handlers} (devono implementare l'interfaccia \texttt{EventHandler})
\item Ogni nodo può registrare uno o più event handlers
\begin{itemize}
\item In generale, attraverso i metodi \texttt{setOn...}()
\item Ogni event handler deve implementare il metodo \texttt{void handle(ActionEvent e)}
\end{itemize}
\end{itemize}

\begin{block}{Es. Gestione del click su un Button Node}
\begin{lstlisting}
btn.setOnMouseClicked(eventHandler -> {
 lbl.setText("Hello, JavaFX World!");
});
\end{lstlisting}
\end{block}
\end{frame}

\begin{frame}[fragile]{Esempio con più Stage (1/2)}
\begin{lstlisting}[basicstyle=\tiny]
public class Main extends Application {

  @Override
  public final void start(final Stage mainStage) {
    final Scene scene = new Scene(initSceneUI());
    mainStage.setScene(scene);
    mainStage.setTitle("JavaFX Example");
    mainStage.show();
  }

  private Parent initSceneUI() {
    final Label inputLbl = new Label("Input: ");
    final TextField inputArea = new TextField();
    final Button okBtn = new Button("Open a new Stage with the input data!");
       
    okBtn.setOnMouseClicked(eventHandler -> {
      new SecondStage(inputArea.getText()).show();
    });

    final BorderPane root = new BorderPane();
    root.setRight(okBtn);
    root.setLeft(inputLbl);
    root.setCenter(inputArea);

    BorderPane.setAlignment(inputLbl, Pos.CENTER_LEFT);
    BorderPane.setAlignment(okBtn, Pos.CENTER_RIGHT);

    return root;
  }

  public static void main(final String[] args) { launch(args); }
}
\end{lstlisting}
\end{frame}

\begin{frame}[fragile]{Esempio con più Stage (2/2)}
\begin{lstlisting}[basicstyle=\tiny]
public class SecondStage extends Stage {
  private Label lbl;
	
  public SecondStage(final String message) {
    super();
		
	setTitle("New Window...");
    	
	setScene(new Scene(initSceneUI(), 400, 200));
		
	lbl.setText(message);
  }
	
  private Parent initSceneUI() {
    lbl = new Label();
		
	FlowPane root = new FlowPane();
	root.setAlignment(Pos.CENTER);
		
	root.getChildren().add(lbl);
		
	return root;
  }
}
\end{lstlisting}
\end{frame}



\section{FXML}

\begin{frame}{Separazioni di ruoli e contenuti}
\begin{itemize}\itemsep10pt
\item In JavaFX è possibile separare il design della GUI dal codice sorgente che la riguarda
\item Il design della GUI può essere descritto attraverso un linguaggio di markup denominato FXML
\end{itemize}
\begin{figure}
\includegraphics[width=0.4\textwidth]{img/soc.png}
\end{figure}
\end{frame}

\begin{frame}{FXML}
\begin{itemize}\itemsep20pt
\item Linguaggio di markup basato su XML
\item Descrive la struttura della GUI
\begin{itemize}
\item Tutti i componenti della GUI sono specificati mediante tag specifici
\item Le proprietà sono specificate come attributi su ciascun tag, nella forma chiave-valore
\end{itemize}
\item Ogni file FXML (con estensione \texttt{.fxml}) deve essere un file XML valido
\begin{itemize}
\item Deve iniziare con il tag: \texttt{<?xml version="1.0" encoding="UTF-8"?>}
\end{itemize}
\end{itemize}
\end{frame}

\begin{frame}[fragile]{Esempio di GUI in FXML}
\begin{lstlisting}
<?xml version="1.0" encoding="UTF-8"?>

<?import javafx.scene.control.*?>
<?import javafx.scene.layout.*?>

<VBox xmlns="http://javafx.com/javafx" xmlns:fx="http://javafx.com/fxml">
  <children>
    <Button fx:id="btn" 
    	alignment="CENTER"
    	text="Say Hello!"
    	textAlignment="CENTER" />
    
    <Label fx:id="lbl"
    	alignment="CENTER_LEFT"
    	text="Label Text Here!"
    	textAlignment="LEFT" />
  </children>
</VBox>
\end{lstlisting}
\end{frame}

\begin{frame}[fragile]{Esempio di GUI in FXML -- Note}
\begin{enumerate}\itemsep15pt
\item Attraverso il tag \texttt{<?import ... ?>} è possibile specificare i package in cui recuperare le classi dei componenti d'interesse
\begin{itemize}
\item E' equivalente all'\texttt{import} di Java
\end{itemize}
\item Il container principale (unico per il singolo file) \underline{deve} specificare gli attributi \texttt{xmlns} e \texttt{xmlns:fx}
\begin{itemize}
\item \begin{verbatim}xmlns="http://javafx.com/javafx"\end{verbatim}
\item \begin{verbatim}xmlns:fx="http://javafx.com/fxml"\end{verbatim}
\end{itemize}
\item Ogni container deve specificare i nodi figli all'interno dei tag \texttt{$<$children$>$} e \texttt{$<$/children$>$}
\item Ogni nodo deve definire il proprio ID mediante l'attributo \texttt{fx:id}
\begin{itemize}
\item Es. \texttt{$<$TextField fx:id="textField1"/$>$}
\end{itemize}
\end{enumerate}
\end{frame}

\begin{frame}[fragile]{Collegare il design della GUI al codice Java}
\begin{itemize}\itemsep10pt
\item La GUI descritta nel file FXML deve essere collegata alla scena agganciata allo stage dell'applicazione
\item Si può utilizzare il componente \texttt{javafx.fxml.FXMLLoader}
\begin{itemize}
\item Il metodo statico \texttt{load(URL location)} 
\end{itemize}
\end{itemize}
\begin{block}{FXMLLoader (esempio)}
\begin{itemize}
\item Si suppone che nel progetto sia presente il file \texttt{main.fxml} contenente una descrizione valida per la GUI da caricare
\end{itemize}
\begin{lstlisting}
Parent root = FXMLLoader.load(ClassLoader.getSystemResource("layouts/main.fxml"));
\end{lstlisting}
\end{block}
\end{frame}

\begin{frame}[fragile]{FXMLLoader (esempio completo)}
\begin{lstlisting}
public class Example3 extends Application{

	@Override
	public void start(Stage stage) throws Exception {
Parent root = FXMLLoader.load(ClassLoader.getSystemResource("layouts/main.fxml"));
		
		Scene scene = new Scene(root, 500, 250);
		
		stage.setTitle("JavaFX - Example 3");
		stage.setScene(scene);
		stage.show();
	}
	
	public static void main(String[] args) {
		launch(args);
	}
}
\end{lstlisting}
\end{frame}

\begin{frame}[fragile]{Lookup dei componenti della GUI}
\begin{itemize}
\item Il riferimento ai componenti (nodi) inseriti nella GUI definita nel file FXML può essere recuperato tramite la scena a cui la GUI è stata collegata
\begin{itemize}
\item Metodo \texttt{Node lookup(String id)}
\end{itemize}

\begin{block}{Node Lookup (esempio)}
\begin{lstlisting}
Label lbl = (Label) scene.lookup("#lbl");
		
Button btn = (Button) scene.lookup("#btn");
btn.setOnMouseClicked(handler -> {
	lbl.setText("Hello, FXML!");
});
\end{lstlisting}
\end{block}
\item \textbf{Attenzione}: il metodo \texttt{lookup} richiede come parametro l'id specificato per il componente (attributo \texttt{fx:id} nel file FXML) preceduto dal simbolo \#
\end{itemize}
\end{frame}

\begin{frame}{GUI Controller e Node Injection}
\begin{itemize}\itemsep20pt
\item Per una corretta separazione dei contenuti (e una buona implementazione del pattern MVC in JavaFX) è opportuno specificare un oggetto \emph{controller} per ciascuna GUI
\begin{itemize}
\item Il parent component della GUI deve definire l'attributo \texttt{fx:controller} con valore riferito al percorso completo della classe che fungerà da controller
\end{itemize}
\item Mediante l'annotazione \texttt{@FXML} è possibile recuperare:
\begin{itemize}
\item I riferimenti ai vari nodi (senza utilizzare il meccanismo di lookup)
\item Associare gli event handler ai vari eventi dei componenti
\end{itemize}
\end{itemize}
\end{frame}

\begin{frame}[fragile]{Esempio Completo (1/3) -- Application}
\begin{lstlisting}
public class CompleteExample extends Application{

	@Override
	public void start(Stage stage) throws Exception {
		VBox root = FXMLLoader.load(ClassLoader.getSystemResource("layouts/main.fxml"));
		
		Scene scene = new Scene(root, 500, 250);
		
		stage.setTitle("JavaFX - Complete Example");
		stage.setScene(scene);
		stage.show();
	}
	
	public static void main(String[] args) {
		launch(args);
	}
}
\end{lstlisting}
\end{frame}

\begin{frame}[fragile]{Esempio Completo (2/3) -- GUI (FXML file)}
\begin{lstlisting}
<?xml version="1.0" encoding="UTF-8"?>

<?import javafx.scene.control.*?>
<?import javafx.scene.layout.*?>

<VBox 
    xmlns="http://javafx.com/javafx"
    xmlns:fx="http://javafx.com/fxml"
	fx:controller="it.unibo.oop.lab.javafx.UIController">
  <children>
    <Button fx:id="btn"
    	alignment="CENTER"
    	text="Say Hello!"
    	onMouseClicked="#btnOnClickHandler" />
    
    <Label fx:id="lbl"
    	alignment="CENTER_LEFT"
    	text="Label Text Here!" />
  </children>
</VBox>
\end{lstlisting}
\end{frame}

\begin{frame}[fragile]{Esempio Completo (3/3) -- GUI Controller}
\begin{lstlisting}
public class UIController {

	@FXML
	private Label lbl;
	
	@FXML
	private Button btn;
	
	@FXML
	public void btnOnClickHandler() {
		lbl.setText("Hello, World!");
	}
}
\end{lstlisting}
\end{frame}

\section{JavaFX integrato in Swing}

\begin{frame}{Integrare JavaFX in Swing}
\begin{itemize}\itemsep20pt
\item In Java, un'applicazione può contenere sia GUI programmate in Swing, sia altre programmate in JavaFX
\begin{itemize}
\item La Main UI deve essere in Swing
\item Si integrano Scene di JavaFX in JFrame avvalendosi di componenti che sono istanze di JFXPanel, eseguite nel thread specifico di JavaFX
\end{itemize}
\item Va prestata particolare attenzione a dove viene eseguito il codice che gestisce la GUI
\begin{itemize}
\item \texttt{javafx.application.Platform.runLater()}, per eseguire codice nel thread dedicato a JavaFX
\item \texttt{javax.swing.SwingUtilities.invokeLater()}, per eseguire codice nel thread dedicato s Swing
\end{itemize}
\end{itemize}
\end{frame}

\begin{frame}[fragile]{Swing e JavaFX: esempio (1/2)}
\begin{lstlisting}[basicstyle=\tiny]
private static void initMainJFrame(final JFrame frame) {
  final JButton button = new JButton();
  button.setText("Launch JavaFX Scene");
  button.addActionListener(event -> {
            
    final JFXPanel jfxPanel = new JFXPanel();
            
    Platform.runLater(() -> {
      jfxPanel.setScene(new Scene(initJavaFXSceneUI(), 300, 300));
                    
      SwingUtilities.invokeLater(() -> {
        final JFrame frameWithJavaFX = new JFrame("JFrame with JavaFX embedded!");
        frameWithJavaFX.add(jfxPanel);
        frameWithJavaFX.pack();
        frameWithJavaFX.setVisible(true);
      });
    });
  });
        
  final JPanel panel = new JPanel();
  panel.setLayout(new FlowLayout());
  panel.add(button);

  frame.setContentPane(panel);
  frame.setSize(300, 300);
  frame.setDefaultCloseOperation(JFrame.EXIT_ON_CLOSE);
  frame.setVisible(true);
}
\end{lstlisting}
\end{frame}

\begin{frame}[fragile]{Swing e JavaFX: esempio (2/2)}
\begin{lstlisting}[basicstyle=\tiny]  
private static Parent initJavaFXSceneUI() {
  final Label lbl = new Label();
  lbl.setText("Hello, JavaFX World!");

  final Button btn = new Button();
  btn.setText("Say Hello");
  btn.setOnMouseClicked(eventHandler -> {
    lbl.setText("Hello from Button!");
  });

  final VBox root = new VBox();
  root.getChildren().add(lbl);
  root.getChildren().add(btn);

  return root;
}
\end{lstlisting}

\begin{lstlisting}[basicstyle=\tiny]
public static void main(final String[] args){
  initMainJFrame(new JFrame("JFrame GUI"));
}
\end{lstlisting}
\end{frame}

\section{Utilizzo di JavaFX in Eclipse}

\subsection{JavaFX con JDK 8}

\begin{frame}{JavaFX in Eclipse (JDK 8)}
\begin{itemize}\itemsep20pt
\item La libreria JavaFX è presente unicamente nel JDK 8
\begin{itemize}
\item Distribuita attraverso il file \texttt{jfxrt.jar} presente nella directory \texttt{/lib/ext} della propria installazione del JDK
\end{itemize}
\item Tutti le librerire \emph{esterne} \underline{non} sono automaticamente accessibili da tutti i progetti aperti nell'IDE Eclipse
\begin{itemize}
\item In realtà dipende dalla versione di Eclipse\dots
\end{itemize}
\item Deve essere definita una regola d'accesso per JavaFX in relazione allo specifico progetto Java aperto in Eclipse
\end{itemize}
\end{frame}

\begin{frame}{Configurazione del progetto (1/2)}
\begin{itemize}
\item (tasto DX sul progetto) $>$ Properties $>$ Java Build Path
\item (tab Libraries) $>$ Click sul JRE System Library $>$ Selezionare \textit{Access rules} $>$ Click su Edit
\end{itemize}
\begin{figure}
\includegraphics[width=0.9\textwidth]{img/conf01.png}
\end{figure}
\end{frame}

\begin{frame}{Configurazione del progetto (2/2)}
\begin{itemize}
\item Click su Add $>$ Aggiungere una regola con:
\begin{itemize}
\item Resolution: \textbf{Accessible}
\item Rule pattern: \textbf{javafx/**}
\end{itemize}
\item OK $>$ OK $>$ Apply and Close 
\end{itemize}
\begin{figure}
\includegraphics[width=0.85\textwidth]{img/conf02.png}
\end{figure}
\end{frame}

\subsection{JavaFX con JDK 11}

\begin{frame}{JavaFX in Eclipse (JDK 11)}
\begin{itemize}\itemsep20pt
\item JavaFX deve essere importato nel progetto come libreria esterna
\item Due alternative:
\begin{enumerate}
\item Si aggiungono tutti i JAR della libreria direttamente nel progetto
\begin{itemize}
\item Scaricabili da \url{https://gluonhq.com/products/javafx/}
\end{itemize}
\item Si specificano le dipendenze via Gradle
\end{enumerate}
\item Oggigiorno, è preferibile optare per la seconda alternativa
\end{itemize}
\end{frame}


\begin{frame}[fragile]{JavaFX e Gradle}
\begin{columns}[t]
\begin{column}{0.3\textwidth}
\begin{figure}
\includegraphics[width=\textwidth]{img/javafx-gradle-project.png}
\end{figure}
\end{column}
\begin{column}{0.7\textwidth}
\begin{block}{\texttt{build.gradle} (sintassi Groovy)}

\begin{lstlisting}[basicstyle=\tiny]
plugins {
    id 'application'
    id 'org.openjfx.javafxplugin' version '0.0.8'
}

repositories {
    mavenCentral()
}

javafx {
    version = "13"
    modules = [ 'javafx.controls', 'javafx.fxml' ]
}

mainClassName = 'application.Main'
\end{lstlisting}

\end{block}
\end{column}
\end{columns}
\end{frame}

\begin{frame}{Esecuzione dell'applicazione da Eclipse IDE}
\begin{itemize}\itemsep20pt
\item Si utilizza il task (predefinito) \emph{run} di Gradle -- accessibile dalla View \emph{Gradle Tasks} di Eclipse
\begin{figure}
\includegraphics[width=0.9\textwidth]{img/gradle-tasks-view.png}
\end{figure}
\end{itemize}
\begin{itemize}
\item Qualora si voglia utilizzare il debugger, o eseguire l'applicazione in \emph{modo tradizionale}, è necessario (solo per lo sviluppo sul singolo computer) specificare alcuni argomenti per la VM nella run configuration
\begin{itemize}
\item \texttt{--module-path \textbackslash path \textbackslash to \textbackslash javafx \textbackslash sdk \textbackslash lib \\--add-modules javafx.controls,javafx.fxml}
\item In questo caso, è necessario avere una copia dell'SDK in locale
\end{itemize}
\end{itemize}
\end{frame}

\begin{frame}[fragile]{Generazione di JAR con JavaFX (1/2)}
\begin{itemize}
\item Abbiamo visto che la \emph{main} class di un progetto JavaFX (vedi linee guida) deve:
\begin{itemize}
\item estendere \texttt{javafx.application.Application} 
\item il metodo main deve chiamare il metodo statico \texttt{launch()} ereditato da \texttt{Application}
\end{itemize}
\item Ad esempio: 
\end{itemize}

\begin{footnotesize}
\begin{lstlisting}
import javafx.application.Application;

public class MyApplication extends Application {

	@Override
	public void start(final Stage stage) throws Exception {
		//TODO
	}
	
	public static void main(final String[] args) {
		launch(args);
	}
}
\end{lstlisting}
\end{footnotesize}
\end{frame}

\begin{frame}{Generazione di JAR con JavaFX (2/2)}
\begin{itemize}
\item Se si tenta di creare un jar eseguibile specificando \texttt{MyApplication} come main class si ottiene un errore quando si va ad eseguire il JAR
\begin{itemize}
\item Da Java 11 -- con l'introduzione dei moduli e per come è implementata \texttt{javafx.application.Application} -- si presenta un conflitto che non consente di eseguire fuori dall'IDE applicazioni JavaFX
\item Dipende per lo più da come la JVM (del JDK 11) gestisce i processi quando manda in esecuzione un JAR
\end{itemize}
\item Serve un workaround...
\begin{itemize}
\item Utilizziamo una classe che funga da \emph{launcher}
\item Richiama il main della classe principale
\end{itemize}
\end{itemize}
\end{frame}

\begin{frame}[fragile]{Classe Launcher}
\begin{lstlisting}
import javafx.application.Application;

public class MyApplication extends Application {

	@Override
	public void start(final Stage stage) throws Exception {
		//TODO
	}
	
	public static void main(final String[] args) {
		launch(args);
	}
}
\end{lstlisting}

\begin{lstlisting}
public class Launcher {
    
    public static void main(String[] args) {
        MyApplication.main(args);
    }
}
\end{lstlisting}
\end{frame}

\begin{frame}[fragile]{JAR Gradle Task}
\begin{itemize}
\item E' opportuno far generare il JAR a Gradle, mediante task dedicato
\begin{lstlisting}
jar {   
    manifest {
        attributes 'Main-Class': 'myapp.Launcher'
    }
    from {
        configurations.runtimeClasspath.collect { 
        	it.isDirectory() ? it : zipTree(it) 
        }
    }
}
\end{lstlisting}
\item Attenzione: sintassi Groovy (da inserire nel file build.gradle)
\end{itemize}
\end{frame}

\begin{frame}[fragile]{\texttt{build.gradle} (completo)}
\begin{lstlisting}[basicstyle=\tiny]
plugins {
  id 'application'
  id 'org.openjfx.javafxplugin' version '0.0.8'
}

repositories {
  mavenCentral()
}

dependencies {
  /* for cross-platform jar: */
  runtimeOnly "org.openjfx:javafx-graphics:$javafx.version:win"
  runtimeOnly "org.openjfx:javafx-graphics:$javafx.version:linux"
  runtimeOnly "org.openjfx:javafx-graphics:$javafx.version:mac"
}

javafx {
  version = "13"
  modules = [ 'javafx.controls', 'javafx.fxml' ]
}

mainClassName = 'application.Main'

jar {
  manifest {
    attributes 'Main-Class': 'application.Launcher'
  }
  
  from {
    configurations.runtimeClasspath.collect { it.isDirectory() ? it : zipTree(it) }
  }
}
\end{lstlisting}
\end{frame}

\subsection{Scene Builder}

\begin{frame}{Scene Builder 2.0}
\begin{itemize}\itemsep10pt
\item Strumento per la creazione di GUI JavaFX-based in modalità drag-n-drop (GUI Builder)
\item Consente di esportare il file FXML relativo alla GUI disegnata
\item Distribuito come strumento esterno al JDK, non integrato (direttamente) in Eclipse
\end{itemize}
\begin{block}{Download Page}
\url{http://www.oracle.com/technetwork/java/javafxscenebuilder-1x-archive-2199384.html}
\begin{itemize}
\item attenzione: scaricare la versione 2.0
\end{itemize}
\end{block}
\end{frame}

\begin{frame}{Scene Builder 2.0}
\begin{figure}
\includegraphics[width=\textwidth]{img/scenebuilder.png}
\end{figure}
\end{frame}

\subsection{Plugin e(fx)clipse}

\begin{frame}{Plugin e(fx)clipse v. 3.0.0}
\begin{figure}
\includegraphics[width=0.8\textwidth]{img/plugin-install.png}
\end{figure}
\begin{itemize}\itemsep5pt
\item Plugin che fornisce una serie di tool per la gestione di applicazioni e progetti java basati su JavaFX
\item Consente di accedere al tool Scene Builder (installato separatamente) direttamente da Eclipse
\item Installabile dall'Eclipse Marketplace
\begin{itemize}
\item link alla pagina del progetto: \url{http://www.eclipse.org/efxclipse/index.html}
\end{itemize}
\item \textbf{Va oltre le esigenze di questo corso...}
\end{itemize}
\end{frame}



%\begin{frame}[allowframebreaks]{Bibliography}
%\bibliography{biblio.bib}
%\end{frame}

\end{document}

